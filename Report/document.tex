
\chapter{Introdução}

  O tema do nosso trabalho é a análise de tendências no \nonport{YouTube}, com foco em compreender quais tipos de vídeos estão em alta na plataforma, identificar os canais mais influentes e observar o engajamento que esses vídeos recebem, medido em termos de curtidas, comentários e visualizações. Através dessa análise, buscamos extrair \nonport{insights} que permitam caracterizar o que define um vídeo popular, os padrões de conteúdo que se destacam, e a relação entre a popularidade e o país de origem dos vídeos. Para isso, estamos nos guiando através das seguintes principais questões: ``Quais categorias de vídeos estão em alta?'', ``Quais tipos de canais produzem o conteúdo mais relevante e visualizado?'', ``Como as tendências variam de acordo com as regiões?'' e ``Qual a relação entre quantidade de comentários e \nonport{likes} e a classificação de um vídeo?''. Essas perguntas, além de norteadoras, nos motivaram a escolher esse tema. Por meio desta análise, buscamos não apenas identificar padrões regionais e globais, mas também entender como os algoritmos da plataforma podem influenciar o comportamento do público e o sucesso de certos tipos de conteúdo.

  Os conjuntos de dados selecionados são dos sites \nonport{GitHub} e \nonport{Social Blade} - embora não tenhamos utilizado diretamente este último para a modelagem, apenas servindo como uma alternativa para pegar mais informações no futuro e comparar alguns dados - que podem ser encontrados através dos links:

  \begin{itemize}
    \item \href{https://www.kaggle.com/datasets/datasnaek/youtube-new}{\nonport{Kaggle}} : Site principal utilizado para a coleta de dados. Nele, o autor disponibilizou a \nonport{API} (através de um link que dá acesso a um repositório no \href{https://github.com/mitchelljy/Trending-YouTube-Scraper}{\nonport{GitHub}}), usada para a coleta dos dados disponíveis no site, a qual utilizamos diretamente para a coleta de dados mais recentes para a nossa aplicação. Esse site apresenta um \nonport{CSV} que contém informações gerais dos vídeos em alta separados por país, condensando todo o tipo de dado que possa ser relevante para a análise. Nossa aplicação utiliza essas informações - sobretudo o ranking do vídeo e a quantidade de visualizações - para mapear que tipo de conteúdo domina o topo das tendências nos diferentes países, além de fazer comparações entre as regiões.
    \item \href{https://socialblade.com/}{\nonport{Social Blade}}: site que utilizamos para obter informações extras e comparar alguns dados obtidos pela \nonport{API}. Há uma quantidade muito grande de informações disponibilizadas pelo site. De maneira geral, ele fornece, em tempo real, todas as informações referentes a qualquer canal registrado no \nonport{YouTube} (como quantidade de inscritos, ganho estimado, classificação, etc). Utilizamos este site como um suporte para a análise dos canais que publicaram os vídeos presente no ``em alta''
  \end{itemize}


\chapter{Datasets}

  Como informado na seção anterior, foram considerados, principalmente, dois conjuntos de dados: um que contém um registro, gerado por \nonport{script} pelo usuário, dos vídeos em alta no \nonport{YouTube} em alguns países; e o da plataforma \nonport{Social Blade}, que mantém uma imensa variedade de métricas de diferentes redes sociais para análise estratégica dos criadores de conteúdo.

\begin{enumerate}
  \item Trending YouTube Video Statistics: Este é o principal conjunto de dados utilizado para a especificação do modelo conceitual, e, consequentemente, de todo o projeto do Banco de Dados. Dele buscamos adquirir a maior parte dos dados (ou todos eles) para o povoamento de nosso sistema. A postagem acerca do \nonport{dataset}, explicando seu intuito, conteúdo e possibilidades de uso, foi feita no \href{https://www.kaggle.com/datasets/datasnaek/youtube-new}{\nonport{Kaggle}}. O \nonport{dataset} em si consiste de arquivos \nonport{CSV} gerados pelo \nonport{script} disponibilizado pelo usuário Mitchell Jolly no repositório \href{https://github.com/mitchelljy/Trending-YouTube-Scraper}{\nonport{Trending-YouTube-Scraper}} de seu \href{https://github.com/mitchelljy}{\nonport{GitHub}}. O \nonport{script} foi feito em \nonport{Python} e faz uso da \href{https://developers.google.com/youtube/v3/docs?hl=pt-br}{\nonport{YouTube Data API v3}} para a consulta online dos dados. Cada arquivo CSV tem informações de até 200 vídeos mais em alta do \nonport{YouTube} em um dado país. Os países inclusos para análise na versão do autor são: EUA, Grã-Bretanha, Índia, Alemanha, Canadá, França, Coreia do Sul, Rússia, Japão, Brasil e México (embora o Brasil não tenha aparecido no conjunto de dados exibido como exemplo no \nonport{Kaggle}). O usuário do \nonport{script}, entretanto, tem liberdade para determinar seus países de interesse, bastando, para isso, a edição de um arquivo de texto contido na pasta em que o \nonport{script} se encontra. Além disso, o usuário deve disponibilizar uma chave da API, que deve ser obtida e ativada no \nonport{Google Cloud Console}. De posse de uma chave válida, só resta colá-la em um outro arquivo \nonport{.txt} na pasta e, finalmente, executar o programa pelo console. O código foi escrito para gerar tabelas com até 200 registros de vídeos em alta, para cada país, com os seguintes campos:

  \begin{itemize}
    \item video\underline{ }id: contém, em cada registro, o \nonport{ID} do vídeo gerado pelo próprio \nonport{YouTube}. Esse \nonport{ID} é utilizado, além de mera caracterização de vídeos únicos na plataforma, para a geração de suas \textit{URL}s, por exemplo. Cada \nonport{ID} é uma sequência alfanumérica de símbolos, sendo mais apropriado modelá-lo como cadeia de caracteres (\nonport{string}) em nosso sistema;
    \item title: refere-se ao título do vídeo em sua língua original (em que foi publicado). Suporta caracteres de \nonport{UTF-8}. É, naturalmente, do tipo textual, sendo, portanto, claramente representável por valores do domínio de \nonport{strings}; 
    \item publishedAt: corresponde à data e hora de publicação do vídeo (com relação ao Tempo Universal Coordenado \nonport{UTC}). O valores desse atributo são resgatados no formato \href{https://www.w3.org/TR/NOTE-datetime}{\nonport{ISO 8601}} e podem ser mapeados para o domínio \nonport{DATETIME} do \nonport{SGBD};
    \item channelId: análogo ao \nonport{ID} gerado para o vídeo, este campo armazena o \nonport{ID} criado pelo site para cada canal, neste caso, o canal que publicou o vídeo. É uma cadeia alfanumérica e, consequentemente, está no domínio de \nonport{strings}; 
    \item channelTitle: também análogo ao título do vídeo, é relativo ao título do canal que postou o determinado vídeo. É do tipo textual e mapeado no tipo \nonport{string};
    \item categoryId: corresponde ao identificador da categoria atribuída ao vídeo pelo canal que o publicou. No \href{https://developers.google.com/youtube/v3/docs/videos?hl=pt-br#snippet.categoryId}{guia de referências da \nonport{API}} consta que é um dado do tipo \nonport{string}, porém, após coleta da listagem de categorias disponíveis, percebemos que todas se tratam de inteiros de 1 a 44 (pulando alguns). Portanto, consideramos que a atribuição ao domínio dos inteiros seja a melhor alternativa;
    \item trending\underline{ }date: contém a data em que o vídeo está em alta, ou seja, equivale à data em que o arquivo \nonport{CSV} foi criado. Está em um formato diferente do utilizado pela \nonport{API} do \nonport{YouTube}, pois é uma informação adicionada, na força bruta, no \nonport{script}. Consiste do ano, dia e mês de coleta dos dados, nesta ordem, separados por ponto final (AA.DD.MM). É naturalmente associável ao domínio \nonport{DATE};
    \item tags: corresponde a uma lista de \nonport{tags} (etiquetas) atribuídas ao vídeo, usadas como auxílio para o algoritmo de recomendações da plataforma. São pequenas porções de texto (ocorrências de \nonport{tag}) separadas por barras verticais (|). É permitido que a \nonport{tag} contenha espaços. Cada \nonport{tag} encaixa-se no domínio \nonport{string};
    \item view\underline{ }count: é o número total de visualizações que o vídeo teve desde sua publicação até o instante da captura dos dados. Consiste de um inteiro não-negativo (logicamente) e, por isso, pode ser modelado por um inteiro longo sem sinal;
    \item likes: é uma métrica numérica, assim como a contagem de visualizações, referente ao número de curtidas (\nonport{likes}) que o vídeo obteve até a coleta dos dados. É inteiro longo maior ou igual a zero;
    \item dislikes: assemelha-se ao campo denominado ``likes''. A partir de \href{https://developers.google.com/youtube/v3/docs/videos?hl=pt-br#statistics.dislikeCount}{dezembro de 2021}, passou a necessitar de autorização do canal detentor do vídeo para poder ser coletado pela \nonport{API}. Dessa maneira, a consideração de inclusão desse dado em nosso sistema, atualmente, é inviável;
    \item comment\underline{ }count: novamente, é um campo numérico que retrata a quantidade de comentários que o vídeo recebeu até a captura dos dados. Pertence ao domínio de inteiros longos sem sinal;
    \item thumbnail\underline{ }link: é a \nonport{URL} da \nonport{thumbnail} (imagem de capa) do vídeo. Por ser um \nonport{link}, que é composto por vários caracteres, alfanuméricos e símbolos especiais, em sequência, escolhemos o tipo \nonport{string} para armazená-los;
    \item comments\underline{ }disabled: indica se o canal que publicou o vídeo optou por desativar a postagem de comentários. Nos arquivos \nonport{CSV}, esse campo é assinalado por ``True'' ou ``False'', devendo, evidentemente, ser tratado como um valor de domínio \nonport{booleano};
    \item  ratings\underline{ }disabled: informa se o canal desativou a opção dos espectadores curtirem (ou descurtirem) o vídeo. Assim como ``comments\underline{ }disabled'', somente assume valores ``True'' e ``False'' e, por isso, é traduzido para valores do espectro \nonport{booleano};
    \item description: é a \href{https://developers.google.com/youtube/v3/docs/videos?hl=pt-br#snippet.description}{descrição} do vídeo fornecida pelo seu publicador na língua original. É um texto de tamanho completamente variável (no máximo 5000 caracteres) e pode conter símbolos da codificação \nonport{UTF-8}. Logo, é do tipo \nonport{string}.
  \end{itemize}

  \item Social Blade: Esta é uma \href{https://socialblade.com/}{plataforma online} (\nonport{site}) que incluímos como apoio para as nossas observações. Podemos caracterizá-la como um sistema de consultoria para produtores de conteúdo de uma multiplicidade de redes sociais (\nonport{YouTube, Twitch, Facebook, TikTok, Twitter} (novo \nonport{X}), \nonport{Instagram}, só para citar algumas) que pretendem entender, dentre outras coisas, como suas postagens e atividades nas redes estão engajando (através da análise, em variados intervalos de tempo, do número de visualizações de seus perfis ou da variação da quantidade de seguidores, por exemplo) e qual a sua relevância em comparação a outros criadores segundo vários critérios. O sistema ainda é capaz de realizar projeções de quantidades de acessos e seguidores de 2 meses a 5 anos no futuro. Certamente, o \nonport{Social Blade} possui um banco de dados extremamente complexo e é capaz de performar inúmeras operações complicadas. Nosso papel com esse conjunto de dados não é extraí-lo e agregá-lo ao nosso esquema (até porque isso significaria um projeto muito além dos nossos atuais propósitos), mas usá-lo como referência para potenciais análises feitas no escopo de nosso modelo, servindo como uma forma confirmação.
\end{enumerate} 


\chapter{Projeto do Banco de Dados}

\section{Modelo Conceitual}

  \vspace{2cm}
  \begin{figure}[H]
    \centering
    \includegraphics[width=1.0\linewidth]{Conceitual.png} 
    \caption{Modelagem conceitual do \nonport{Dataset} utilizado.}
  \end{figure}

  Como a base de dados tem os vídeos em alta como foco, as entidades do modelo correspondem aos conceitos relevantes para os vídeos do \nonport{YouTube} e seus dados. A entidade principal é ``Vídeo', sendo cada um identificado por um \nonport{ID} e publicado por um único ``Canal'', nome dado às contas através das quais os produtores de conteúdo publicam suas criações. Cada canal possui um título, pelo qual é conhecido por seu público na plataforma, e um \nonport{ID}, que o identifica para os sistemas do \nonport{YouTube}. Além disso, cada vídeo é, no momento da publicação, associado a uma categoria por seu criador, que assume um dentre 15 valores pré-determinados pela plataforma, como ``Gaming'' ou ``Comedy'', por exemplo.

  As informações de categoria e canal de um determinado vídeo são fixas, e não mudam com o tempo. No entanto, a maioria dos dados sobre um vídeo, como quantidade de \nonport{likes} e visualizações, são constantemente atualizadas, sendo cruciais para o registro da sua popularização. Para representar os dados de um vídeo como estavam no momento em que entrou em alta em um certo país, é incluída a entidade ``Vídeo Snapshot'', que representa um ``fotografia'' do estado do vídeo em uma data. A \nonport{snapshot} possui os dados do vídeo que se referem a um momento do tempo, como número de \nonport{likes}, \nonport{dislikes} e comentários. E, como um canal tem o direito de editar título e descrição de seus vídeos após a publicação, esses dados também são considerados dependentes do tempo, e, portanto, são modelados como atributos da \nonport{snapshot}, e não da entidade vídeo. É importante ressaltar que, como um mesmo vídeo pode ficar em alta durante vários dias, consecutivos ou não, cada um pode se associar a várias \nonport{snapshots}, cada uma referente a um data diferente. Finalmente, para representar que um vídeo esteve em alta em um dado país, é incluída a relação ``Trend'' entre ``País'' e ``Snapshot''.

  \todo{Apresentar e descrever as consultas realizadas. Cada consulta deve ser definida através de uma explicação textual acompanhada do respectivo comando SQL.}

\section{Modelo Lógico}

\section{Modelo Físico}


\chapter{Aplicação}

  \todo{Descrever a aplicação web desenvolvida, explicando como foi implementada e detalhando, com exemplos, as funcionalidades que ela disponibiliza.}


\chapter{Distribuição do trabalho}

  \todo{Indicar explicitamente as atividades realizadas por cada membro do grupo.}


\chapter{Considerações finais}

  \todo{Avaliar os resultados alcançados de acordo com os objetivos propostos.}
